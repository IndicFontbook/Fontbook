\chapter{Open font format}

\section{Introduction}
\section {Opentype feature Tags}
The OpenType format defines a number of typographic features that a particular font may support.
The following tables list the features defined in version 1.6 of the OpenType specification and
applicable for Indic scripts.

OpenType features may be applicable only to certain language scripts or specific languages, 
or in certain writing modes. The features are split into several tables accordingly.

\begin{center}
\begin{longtable}{ | p{2cm} | p{1cm} | p{3cm} | p{5cm} |}
\caption{Opentype feature tags for Indic scripts}
\label{tab:otscripttagsindic} \\

\hline
Long name                   & tag  & type & Description                                                                          \\
\hline
Above-base Forms            & abvf & simple substitution of one glyph with another  & Replaces the diacritic part of a vowel sign in Khmer etc., e.g. ä to aͤ              \\ \hline
Above-base Mark Positioning & abvm & positioning of mark glyphs relative to base or ligature & Positions a diacritic mark on top of the base glyph                                  \\\hline
Above-base Substitutions    & abvs & ligatures   & Replaces a pair of base and top diacritic mark by a ligature, e.g. ä to æ            \\\hline
Below-base Forms            & blwf & ligatures   & Replaces the subscript part of a consonant compound in Khmer etc., e.g. ş to ș       \\\hline
Below-base Mark Positioning & blwm & positioning of mark glyphs relative to base or ligature & Positions a diacritic mark on top of the base glyph                                  \\\hline
Below-base Substitutions    & blws & ligatures  & Replaces a pair of base and bottom diacritic mark by a ligature, e.g. ç to cz        \\\hline
Pre-base Forms              & pref & ligatures   & Khmer and other similar scripts: Myanmar, Malayalam, Telugu                          \\\hline
Pre-base Substitutions      & pres & ligatures and contextual substitution & Indic                                                                                \\\hline
Post-base Substitutions     & psts & ligatures   & Indic (any alphabetic?)                                                              \\\hline
Post-base Forms             & pstf & ligatures   & Khmer and Gurmukhi, Malayalam                                                        \\\hline
Distance                    & dist & positioning of pair of glyphs  & Adjusts horizontal positioning between glyphs                                        \\\hline
Akhand                      & akhn & ligatures  & Hindi for unbreakable, forms CCV ligatures from two consecutive CV glyphs            \\\hline
Halant Forms                & haln & ligatures   & Uses halant forms of CV glyphs, indicating that it is read C, may include virama     \\\hline
Half Form                   & half & ligatures  & Uses half-forms of CV glyphs, indicating that it is read as just C                   \\\hline
Nukta Forms                 & nukt & ligatures & Add nukta (dot mark) to glyph, although this is available through Unicode characters \\\hline
Rakar Forms                 & rkrf & ligatures   & Indic rakar                                                                          \\\hline
Reph Form                   & rphf & ligatures & The reph diacritic changes a CV glyph to its respective rCV glyph                    \\\hline
Vattu Variants              & vatu & ligatures  & Indic vattu                                                                          \\\hline
Conjunct Forms              & cjct & ligatures  &                                                                                      \\\hline
Conjunct Form After Ro      & cfar & simple substitution of one glyph with another   & Khmer                                                                                \\\hline
\end{longtable}
\end{center}
\section{GPOS}

GPOS or Glyph POSitioning table.

[See http://fontforge.org/gposgsub.html\#opentype]

\section{GSUB}

GUSB or Glyph SUBstitution table.

[See http://fontforge.org/gposgsub.html\#opentype]

\section{GDEF}

\section{Shaping Engine}

[Some content from http://behdad.org/text/ can be used ]

\section{Shape Glyph sequence}

\section{Position Glyph sequence}

\section{Reference fonts}

To illustrate the concepts in this document, we will be using a set of
reference fonts for each script, some times more than one per script.

List of fonts table

\section{Reference Rendering Engine}

We are using Harfbuzz as a reference rendering engine. Examples given are
working perfectly with Harfbuzz but any rendering engine conforming to the
Open font specification will give same result.
