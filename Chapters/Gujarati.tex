\chapter{Gujarati}
\secstar{Introduction}

Gujarati script is an abugida like many other Indic scripts rather than
alphabet. It is used for languages like Gujarati and Kutchi. Gujarati script is
variant of Devanagari script differentiated by the loss of the characteristic
horizontal line running above the letters and by a small number of
modifications in the remaining characters.

The modern Gujarati alphabet has 13 vowel letters, 36 consonant letters, 12
vowel extensions and few other symbols.

\secstar{Reference fonts}

We will use Lohit Gujarati as reference font as of now. Other popular Unicode
Gujarati fonts are: Shruti (non free, Microsoft), Rekha, Aakar and Kalapi.

\subsection{Lohit Gujarati}
// FIXME: Short introduction, designers, maintainers, usage info, popularity of
the font.

Lohit Gujarati font is consider as most popular Gujarati font.

\subsection{History}
In 2004, Red Hat released five Indian language fonts as open source licensed
under the GPL. In 2011, Red Hat relicensed fonts under SIL OFL 1.1 license.
The fonts named Lohit which means Red in Sanskrit. Currently, the font family
supports 21 Indian languages: Assamese, Bengali, Devanagari (Hindi, Kashmiri,
Konkani, Maithili, Marathi, Nepali, Sindhi, Santali, Bodo, Dogri), Gujarati,
Kannada, Malayalam, Manipuri, Oriya, Punjabi, Tamil, and Telugu.

Now, Fedora Project and its contributors took the responsibility to consolidate
the further efforts and improvements of the Lohit fonts.

Homepage: {\url{https://fedorahosted.org/lohit/}}

