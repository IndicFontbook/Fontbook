\chapter{General Concepts}

\secstar{Complex script }

Complex script is writing system in which the shape or positining of a character
depends on its relation to other characters.

What makes it complex?
* Bi-directional: text is written right-to-left (For example: Arabic, Hebrew)
and left-to-right (For example: Devanagari)

* Contex sensitive shaping and ligatures: Character may change shape depending
upon position.

* Ordering:
In Gujarati, Ki is where 'i' is place before 'K'.

//FIXME
What is a complex script? What makes it complex, some examples, screenshots
2-3 paragraph + images
How it differs from simple scripts like Latin Character

\secstar{Glyph }
A glyph is an element of writing. By defination glyph is an individual mark on
a written medium that contributes to the meaning of what is written.

Glyph defination also vary from lagnuage to language.

\begin{figure}[hb]
\centering
\includegraphics[width=4in]{Images/glyph.png}
\caption[Glyph example]
{Image of Gujarati 'Aa' showing example of Glyph}
 \end{figure}

\secstar{Ligatures }

A ligature is two or more letters joint together as a single glyph.

\begin{figure}[hb]
\centering
\includegraphics[width=4in]{Images/ligature.png}
\caption[Ligature example]
{Image of example of ligatures}
 \end{figure}

//FIXME
How to identify a ligature Virama
AKA - Pulli, Chandarakkala, Vattu, Halant

\secstar{Cluster/Syllable }

\secstar{Akhand }

\secstar{Ra Forms }
//FIXME
Explain Reph
Explain Rakar

\subsection*{Matra}
//FIXME
Explain prebase, postbase matra with examples and images

\secstar{Split Matra }

\secstar{Reordering }

\secstar{Zero Widh Joiner }

\secstar{Zero Width Non Joiner }

\secstar{Stacking }

