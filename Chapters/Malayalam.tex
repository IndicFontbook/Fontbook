\chapter{Malayalam}
\secstar{Introduction}

Like many other Indic scripts, it is an abugida, or a writing system
that is partially “alphabetic” and partially syllable-based. The
modern Malayalam alphabet has 13 vowel letters, 36 consonant letters,
and a few other symbols. The Malayalam script is a Vattezhuttu script,
which had been extended with Grantha script symbols to represent
Indo-Aryan loanwords. The script is also used to write several
minority languages such as Paniya, Betta Kurumba, and Ravula. The
Malayalam language itself was historically written in several
different scripts.

As is the case for many other Brahmi-derived scripts in the Unicode
Standard, Malayalam uses a virama character to form consonant
conjuncts. The virama sign itself is known as candrakala({\meera ചന്ദ്രക്കല}) in
Malayalam.

When the candrakala sign is visibly shown in Malayalam, it usually
indicates the suppression of the inherent vowel, but it sometimes
indicates instead a reduced schwa sound, often called “half-u” or
samvruthokaram. In the later case, there can also be a -u vowel sign,
and the base character can be a vowel letter. In all cases, the
candrakala sign is represented by the character U+0D4D malayalam sign
virama, which follows any vowel sign that may be present and precedes
any anusvara that may be present.

FIXME samvruthokaram needs more explanation, refer
{\url{http://thottingal.in/documents/Malayalam_Unicode_Report_of_Workshop_Kerala_University.pdf}}
and {\url{http://smc.org.in/doc/rachana-malayalam-collation.pdf}}

\secstar{Orthography variation}

Malayalam has two orthography variations, both actively
used. Generally they are known as Old orthography and Modern
orthography. Old orthography is also known as traditional
orthography. Modern orthography is also known as reformed orthography.

Old orthography is generally identified by large number of ligature
glyphs, often formed by more than one consonants or vowel
combination. For example, in old orthography, consonant {\meera ക} +
vowel { \meera ൂ} will form കൂ as a single ligature. similarly a
conjunct {\meera ക + ് + ത} will form {\meera ക്ത} as a single
ligature. Font developers of traditional stye fonts design and draw
this large set of glyphs. For a reference font Meera, the number of
glyphs is approximately over a thousad.

In the 1970s and 1980s, Malayalam underwent orthographic reform due to
printing difficulties. The treatment of the combining vowel signs u
and uu was simplified at this time. These vowel signs had previously
been represented using special cluster graphemes where the vowel signs
were fused beneath their consonants, but in the reformed orthography
they are represented by spacing characters following their consonants.

// FIXME: insert image here to illustrate both orthography
variation. Refer chapter 9 of Unicode standard page 307 for a sample
image

\subsection {Quasi Vowel Consonants}
In Malayalam some consonants like YA({\meera യ}), RA({\meera ര}) are
quasi-vowels and they have post(pre)/below base forms. As per OT spec the
pstf(pref) and blwf features are always applied and this leads to weird shaping
when some consonants come as base. Eg. {\meera യ്ര, യ്ല, ര്ര, ല്ര}. This is more
prominent in reformed scripts because one can do away with these forms in
traditional scripts.

\secstar{Reference fonts}

Since we need to illustrate both orthographies we will be using 2
fonts. For old orthography, we will use Meera font and for modern
orthography we will use Lohit Malayalam.

\subsection {Meera Font}
{\meera മീര മാതൃക }
// FIXME: Short introduction, designers, maintainers, usage info, popularity of the font.

Meera font is maintained by Swathanthra Malayalam Computing initiative.

Homepage: {\url{https://savannah.nongnu.org/projects/smc}}

\subsection {Lohit Malayalam Font}
// FIXME: Short introduction, designers, maintainers, usage info, popularity of the font.

\subsection {History}
In 2004, Red Hat released five Indian language fonts as open source licensed
under the GPL. In 2011, Red Hat relicensed fonts under SIL OFL 1.1 license.
The fonts named Lohit which means Red in Sanskrit. Currently, the font family
supports 21 Indian languages: Assamese, Bengali, Devanagari (Hindi, Kashmiri,
Konkani, Maithili, Marathi, Nepali, Sindhi, Santali, Bodo, Dogri), Gujarati,
Kannada, Malayalam, Manipuri, Oriya, Punjabi, Tamil, and Telugu.

Now, Fedora Project and its contributors took the responsibility to consolidate
the further efforts and improvements of the Lohit fonts.

Homepage: {\url{https://fedorahosted.org/lohit/}}
