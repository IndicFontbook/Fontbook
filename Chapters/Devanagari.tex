\chapter{Devanagari}
\secstar{Introduction}

Devanagari is an abugida alphabet for India and Nepal. Devanagari is used to
write Standard Hindi, Marathi, Nepali along with Awadhi, Bodo, Bhojpuri,
Gujari, Pahari, (Garhwali and Kumaoni), Konkani, Magahi, Maithili, Marwari,
Bhili, Newar, Santhali, Tharu, and sometimes Sindhi, Dogri, Sherpa, Kashmiri
and Punjabi. It was formerly used to write Gujarati. Because it is the
standardised script for the Hindi language, Devanagari is one of the most used
and adopted writing systems in the world.

\secstar{Reference fonts}

\subsection{Lohit Devanagari}
Lohit Devanagari font is consider as most popular Devanagari font in India.
Gargi, Chandas, Kalimati, Samanata, Nakula are other popular free and
opensource fonts available.

We will take Lohit Devanagari as reference font.

\subsection{History}
In 2004, Red Hat released five Indian language fonts as open source licensed
under the GPL. In 2011, Red Hat relicensed fonts under SIL OFL 1.1 license.
The fonts named Lohit which means Red in Sanskrit. Currently, the font family
supports 21 Indian languages: Assamese, Bengali, Devanagari (Hindi, Kashmiri,
Konkani, Maithili, Marathi, Nepali, Sindhi, Santali, Bodo, Dogri), Gujarati,
Kannada, Malayalam, Manipuri, Oriya, Punjabi, Tamil, and Telugu.

Now, Fedora Project and its contributors took the responsibility to consolidate
the further efforts and improvements of the Lohit fonts.

Homepage: {\url{https://fedorahosted.org/lohit/}}
