\chapter{Bengali}
\section{Introduction}

\section{Reference fonts}

\subsection{Lohit Bengali}

Lohit Bengali is one of the popular fonts for Bengali.

\subsection{History}
In 2004, Red Hat released five fonts for the Indian language under GPL 2.
The fonts were originally developed by Modular Infotech
\footnote{Modular Infotech \url{http://www.modular-infotech.com/}}.
In 2011, Red Hat relicensed fonts under SIL OFL 1.1 license
\footnote{License change announcement of Lohit fonts
\url{https://www.redhat.com/archives/lohit-devel-list/2011-September/msg00000.html}}.
The fonts named Lohit which means Red in Sanskrit. Currently, the font family
supports 21 Indian languages: Assamese, Bengali, Devanagari (Hindi, Kashmiri,
Konkani, Maithili, Marathi, Nepali, Sindhi, Santali, Bodo, Dogri), Gujarati,
Kannada, Malayalam, Manipuri, Oriya, Punjabi, Tamil, and Telugu.

Now, Fedora Project and its contributors took the responsibility to consolidate
the further efforts and improvements of the Lohit fonts.

Homepage: {\url{https://fedorahosted.org/lohit/}}
