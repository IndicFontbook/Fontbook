\chapter{Introduction}

%\begin{chapquote}{Author's name, \textit{Source of this quote}}
%``This is a quote and I don't know who said this.''
%\end{chapquote}

\secstar{Need for document}

In present era if we simply do the comparison between availability of Latin and Indian fonts one can easily observe so much different. only handful fonts available for Indian language compare to thousands or even more fonts and style available for Latin script.

\secstar{Objectives of document}

\secstar{Target Audience for this document}

\secstar{Scope}
\begin{itemize}
\item This document covers only languages and scripts recognized in India.
\item This document covers all complex OFF feature required for Indian scripts.
\item This document is based on the current expertise of community members working/worked on this area.
\item This document is from the typography perspective for each script and many not be linguistically correct.
\item This document does not cover design or calligraphy style aspects but covers only technical aspects.
\item This document is based on Unicode 6.2 and ISO/IEC 14496-22:2009 (Second Edition) "Open Font Format" standard.
\item This document is not a tutorial on font design or development and does not teach typography.
\end{itemize}

\secstar{How to use this document}

Elaborate, Illustrated font design guideline document for Indic fonts by font designers, developers, language experts.
- designer freedom to adapt

\secstar{Notes on Collaboration}

