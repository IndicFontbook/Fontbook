\chapter{Introduction}

\epigraph{
"I can't go to a restaurant and order food because I keep looking at the fonts 
on the menu.-Donald Knuth"
}

One of the integral building blocks for providing multilingual support for 
digital content are fonts. In current times, OpenType fonts are the choice. With 
the increasing need for supporting languages beyond the Latin script, the 
TrueType font specification was extended to include elements for the more 
elaborate writing systems that exist. This effort was jointly undertaken in the 
1990s by Microsoft and Adobe. The outcome of this effort was the OpenType 
Specification - a successor to the TrueType font specification.

Fonts for Indic languages had traditionally been created for the printing 
industry. The TrueType specification provided the baseline for the digital fonts 
that were largely used in desktop publishing. These fonts however suffered from 
inconsistencies arising from technical shortcomings like non-uniform character 
codes. These shortcomings made the fonts highly unreliable for digital content 
and their use across platforms.  The problems with character codes were largely 
alleviated with the gradual standardization through modification and adoption of 
Unicode character codes. The OpenType Specification additionally extended the 
styling and behavior for the typography.

The availability of the specification eased the process of creating Indic 
language fonts with consistent typographic behavior as per the script's 
requirement. However, disconnects between the styling and technical 
implementation hampered the font creation process. Several well-stylized fonts 
were upgraded to the new specification through complicated adjustments, which at 
times compromised on their aesthetic quality.  On the other hand, the technical 
adoption of the specification details was a comparatively new know-how for the 
font designers. To strike a balance, an initiative was undertaken by the a group 
of font developers and designers to document the knowledge acquired from the 
hands own experience for the benefit of upcoming developers and designers in 
this field.

The outcome of the project will be an elaborate, illustrated guideline for font 
designers. A chapter will be dedicated to each of the Indic scripts - Bengali, 
Devanagari, Gujarati, Kannada, Malayalam, Odia, Punjabi, Tamil and Telugu. The 
guidelines will outline the technical representation of the canonical aspects of 
these complex scripts. This is especially important when designing for complex 
scripts where the shape or positioning of a character depends on its relation to 
other characters.

This project is open for participation and contributors can commit directly on 
the project repository.

\section{Objectives of document}

\section{Target Audience for this document}

\section{Scope}
\begin{itemize}
\item This document covers only languages and scripts recognized in India.
\item This document covers all complex OFF features required for Indian scripts.
\item This document is based on the current expertise of community members 
working in this area.
\item This document is from the typography perspective for each script and many 
not be linguistically correct.
\item This document does not cover design or calligraphy style aspects but 
covers only technical aspects.
\item This document is based on Unicode 6.2 and ISO/IEC 14496-22:2009 (Second 
Edition) "Open Font Format" standard.
\item This document is not a tutorial on font design or development and does not 
teach typography.
\end{itemize}

\section{How to use this document}

Elaborate, Illustrated font design guideline document for Indic fonts by font 
designers, developers, language experts.
- designer freedom to adapt

\section{Notes on Collaboration}
