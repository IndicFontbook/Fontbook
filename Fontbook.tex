\documentclass[11.5pt,twoside]{book}
\usepackage[T1]{fontenc}
\usepackage[utf8]{inputenc}
\usepackage{lmodern}
\usepackage{fontbook}


\usepackage[colorlinks=false, pdfborder={0 0 0}, pdftitle={Open Font Format Specification for Scripts of India},
  pdfauthor={Santhosh Thottingal}, pdfsubject={Open Font Format Specification}, pdfkeywords={Font, indic, open , scripts , india}]{hyperref}

%---------------------------------------------------

\begin{document}
%% Set mininum characters on left and right for hyphenation
\lefthyphenmin=3
\righthyphenmin=5
%\normalsize=11.5pt

%-----------------------------Cover Page----------------------------------------

%\includepdf{cover}

%------------------------------Title Page --------------------------------------
\begin{center}

\vspace*{2 in}

\HRule \\[0.4cm]
{\huge \bfseries Open Font Format Specification for Scripts of India}
\\
{\small \bfseries Elaborate, Illustrated font development guideline document for scripts of India by font designers, developers, language experts.}
\\
\HRule \\[0.4cm]

\emph{Author}\\
Santhosh \textsc{Thottingal}


\vfill

{\large \today}

\end{center}
%------------------------------Dedication --------------------------------------
\frontmatter
~
\vfill

%\centerline{{\dejavu ♡}Copyheart 2012. Copying is an act of love. Love is not subject to law. Please copy and share.}
%\centerline{{\dejavu ♡}പകര്‍പ്പിഷ്ടം 2012.  ഇഷ്ടപ്പെടുന്നവരാണു് പകര്‍ത്തുന്നതു്. ഇഷ്ടത്തിനു് നിയമമില്ല. പകര്‍ത്തുവിന്‍ പങ്കുവെയ്ക്കുവിന്‍ }
\begin{center}Dedicated to all font designers\end{center}


\vfill
~
\newpage




%%%%%%%%%%%%%%%%%%%%%%%%%%%%%%%%%%%%%%%%%%%%%%%%%%%%%%%%%%%%%%%%%%%%%%%%
% Auto-generated table of contents, list of figures and list of tables %
%%%%%%%%%%%%%%%%%%%%%%%%%%%%%%%%%%%%%%%%%%%%%%%%%%%%%%%%%%%%%%%%%%%%%%%%
\tableofcontents
%\listoffigures
%\listoftables

\mainmatter

%%%%%%%%%%%
% Preface %
%%%%%%%%%%%
\chapter{Introduction}

\begin{chapquote}{Author's name, \textit{Source of this quote}}
``This is a quote and I don't know who said this.''
\end{chapquote}

\secstar{Need for document}

In present era if we simply do the comparison between availability of Latin and Indian fonts one can easily observe so much different. only handful fonts available for Indian language compare to thousands or even more fonts and style available for Latin script.

\secstar{Objectives of document}

\secstar{Target Audience for this document}

\secstar{Scope}
\begin{itemize}
\item This document covers only languages and scripts recognized in India.
\item This document covers all complex OFF feature required for Indian scripts.
\item This document is based on the current expertise of community members working/worked on this area.
\item This document is from the typography perspective for each script and many not be linguistically correct.
\item This document does not cover design or calligraphy style aspects but covers only technical aspects.
\item This document is based on Unicode 6.2 and ISO/IEC 14496-22:2009 (Second Edition) "Open Font Format" standard.
\item This document is not a tutorial on font design or development and does not teach typography.
\end{itemize}

\secstar{How to use this document}

Elaborate, Illustrated font design guideline document for Indic fonts by font designers, developers, language experts.
- designer freedom to adapt

\secstar{Notes on Collaboration}

\chapter{General Concepts} 

\secstar{Complex script }

[What is a complex script? What makes it complex, some examples, screenshots]
2-3 paragraph + images
How it differs from simple scripts like Latin
Character

\secstar{Glyph }

\secstar{Ligatures }

What is a Ligature, example, image
How to identify a ligature
Virama

AKA - Pulli, Chandarakkala, Vattu, Halant

\secstar{Cluster/Syllable }


\secstar{Akhand }

\secstar{Ra Forms }

	Explain Reph
Explain Rakar
\subsection*{Matra}

[explain prebase, postbase matra with examples and images]
\secstar{Split Matra }

\secstar{Reordering }

\secstar{Zero Widh Joiner }

\secstar{Zero Width Non Joiner }


\secstar{Stacking }

\chapter{Open font format}

\secstar{Introduction}

\secstar{GPOS}

\secstar{GSUB}

\secstar{GDEF}

\secstar{Shaping Engine}

[Some conent from http://behdad.org/text/ can be used ]
\secstar{Shape Glyph sequence}

\secstar{Position Glyph sequence}

\secstar{Reference fonts}

To illustrate the concepts in this document, we will be using a set of reference fonts for each script, some times more than one per script. 
List of fonts table

\secstar{Reference Rendering Engine}

We are using Harfbuzz as  a reference rendering engine. Examples given are working perfectly with Harfbuzz but any rendering engine conforms to the Open font specification will give same result.


\chapter{Devanagari}
\chapter{Tamil}

\chapter{Kannada}

\chapter{Panjabi}

\chapter{Telugu}

\chapter{Gujarati}

\chapter{Odiya}

\chapter{Malayalam}
\section{Introduction}

Like many other Indic scripts, it is an abugida, or a writing system that is partially “alphabetic” and partially syllable-based. The modern Malayalam alphabet has 13 vowel letters, 36 consonant letters, and a few other symbols. The Malayalam script is a Vattezhuttu script, which had been extended with Grantha script symbols to represent Indo-Aryan loanwords. The script is also used to write several minority languages such as Paniya, Betta Kurumba, and Ravula. The Malayalam language itself was historically written in several different scripts.

As is the case for many other Brahmi-derived scripts in the Unicode Standard, Malayalam uses a virama character to form consonant conjuncts. The virama sign itself is known as candrakala(ചന്ദ്രക്കല) in Malayalam. 

When the candrakala sign is visibly shown in Malayalam, it usually indicates the suppression of the inherent vowel, but it sometimes indicates instead a reduced schwa sound, often called “half-u” or samvruthokaram. In the later case, there can also be a -u vowel sign, and the base character can be a vowel letter. In all cases, the candrakala sign is represented by the character U+0D4D malayalam sign virama, which follows any vowel sign that may be present and precedes any anusvara that may be present. 

FIXME samvruthokaram needs more explanation, refer {\url{http://thottingal.in/documents/Malayalam_Unicode_Report_of_Workshop_Kerala_University.pdf}} and {\url{http://smc.org.in/doc/rachana-malayalam-collation.pdf}}

\section{Orthography variation}

Malayalam has two orthography variations, both actively used. Generally they are known as Old orthography and Modern orthography. Old orthography is also known as traditional orthography. Modern orthography is also known as reformed orthography.

Old orthography is generally identified by large number of ligature glyphs, often formed by more than one consonants or vowel combination. For example, in old orthography, consonant {\meera ക} + vowel { \meera ൂ}  will  form കൂ as a single ligature. similarly a conjunct {\meera ക + ് + ത} will form {\meera ക്ത} as a single ligature. Font developers of traditional stye fonts design and draw this large set of glyphs. For a reference font Meera, the number of glyphs is approximately over a thousad.

In the 1970s and 1980s, Malayalam underwent orthographic reform due to printing difficulties. The treatment of the combining vowel signs u and uu was simplified at this time. These vowel signs had previously been represented using special cluster graphemes where the vowel signs were fused beneath their consonants, but in the reformed orthography they are represented by spacing characters following their consonants.

// FIXME: insert image here to illustrate both orthography variation. Refer chapter 9 of Unicode standard page 307 for a sample image

\section{Reference fonts}

Since we need to illustrate both orthographies we will be using 2 fonts. For old orthography, we will use Meera font and for modern orthography we will use Lohit Malayalam.

\subsection{Meera Font}
// FIXME: Short introduction, designers, maintainers, usage infor, popularity of the font

\subsection{Lohit Malayalam Font}
// FIXME: Short introduction, designers, maintainers, usage infor, popularity of the font




\chapter{Glossary}


\chapter{References}


State of Text Rendering By Behdad Esfahbod <behdad behdad com>   http://behdad.org/text/ 


\chapter{Appendices}

[Editor notes]

\end{document}
