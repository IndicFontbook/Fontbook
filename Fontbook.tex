\documentclass[11.5pt,twoside]{book}
\usepackage{graphicx}
\usepackage{fontbook}
\usepackage[colorlinks=false,
  pdfborder={0 0 0},
  pdftitle={A Guide to Developing Fonts for Indian Languages},
  pdfauthor={Santhosh Thottingal},
  pdfsubject={A Guide to Developing Fonts for Indian Languages},
  pdfkeywords={Font, indic, open , scripts , india}
  ]{hyperref}

%---------------------------------------------------

\begin{document}
%% Set mininum characters on left and right for hyphenation
\lefthyphenmin=3
\righthyphenmin=3
\hyphenpenalty=0

%\normalsize=11.5pt

%-----------------------------Cover Page----------------------------------------

%\includepdf{cover}

%------------------------------Title Page --------------------------------------
\begin{center}

\vspace*{2 in}

\HRule \\[0.4cm]
{\huge \bfseries \color{cyan} A Guide To Indian Language Font Development}
\\[1cm]
{\small \bfseries Elaborate, Illustrated font development guideline document
for scripts of India by font designers, developers, language experts.}
\\
\HRule \\[0.4cm]

\emph{Authors}\\
Santhosh \textsc{Thottingal}\\
Your name


\vfill

{\large \today}\\
\url{https://github.com/IndicFontbook/Fontbook}

\end{center}
\thispagestyle{empty}
%%%%%%%%%%%%%%%%%%%%%%%%%%%%%%%%%%%%%%%%%%%%%%%%%%%%%%%%%%%%%%%%%%%%%%%%
% -------------------------Dedication----------------------------------%
%%%%%%%%%%%%%%%%%%%%%%%%%%%%%%%%%%%%%%%%%%%%%%%%%%%%%%%%%%%%%%%%%%%%%%%%
\frontmatter
~
\vfill
\begin{center}Dedicated to all font designers\end{center}
\vfill
~
\thispagestyle{empty}
\newpage
%%%%%%%%%%%%%%%%%%%%%%%%%%%%%%%%%%%%%%%%%%%%%%%%%%%%%%%%%%%%%%%%%%%%%%%%
% Auto-generated table of contents, list of figures and list of tables %
%%%%%%%%%%%%%%%%%%%%%%%%%%%%%%%%%%%%%%%%%%%%%%%%%%%%%%%%%%%%%%%%%%%%%%%%
\tableofcontents
\thispagestyle{empty}
%\listoffigures
%\listoftables

\mainmatter
%%%%%%%%%%%%
% Chapters %
%%%%%%%%%%%%
\chapter{Introduction}

\epigraph{
"I can't go to a restaurant and order food because I keep looking at the fonts 
on the menu.-Donald Knuth"
}

One of the integral building blocks for providing multilingual support for 
digital content are fonts. In current times, OpenType fonts are the choice. With 
the increasing need for supporting languages beyond the Latin script, the 
TrueType font specification was extended to include elements for the more 
elaborate writing systems that exist. This effort was jointly undertaken in the 
1990s by Microsoft and Adobe. The outcome of this effort was the OpenType 
Specification - a successor to the TrueType font specification.

Fonts for Indic languages had traditionally been created for the printing 
industry. The TrueType specification provided the baseline for the digital fonts 
that were largely used in desktop publishing. These fonts however suffered from 
inconsistencies arising from technical shortcomings like non-uniform character 
codes. These shortcomings made the fonts highly unreliable for digital content 
and their use across platforms.  The problems with character codes were largely 
alleviated with the gradual standardization through modification and adoption of 
Unicode character codes. The OpenType Specification additionally extended the 
styling and behavior for the typography.

The availability of the specification eased the process of creating Indic 
language fonts with consistent typographic behavior as per the script's 
requirement. However, disconnects between the styling and technical 
implementation hampered the font creation process. Several well-stylized fonts 
were upgraded to the new specification through complicated adjustments, which at 
times compromised on their aesthetic quality.  On the other hand, the technical 
adoption of the specification details was a comparatively new know-how for the 
font designers. To strike a balance, an initiative was undertaken by the a group 
of font developers and designers to document the knowledge acquired from the 
hands own experience for the benefit of upcoming developers and designers in 
this field.

The outcome of the project will be an elaborate, illustrated guideline for font 
designers. A chapter will be dedicated to each of the Indic scripts - Bengali, 
Devanagari, Gujarati, Kannada, Malayalam, Odia, Punjabi, Tamil and Telugu. The 
guidelines will outline the technical representation of the canonical aspects of 
these complex scripts. This is especially important when designing for complex 
scripts where the shape or positioning of a character depends on its relation to 
other characters.

This project is open for participation and contributors can commit directly on 
the project repository.

\section{Objectives of document}

\section{Target Audience for this document}

\section{Scope}
\begin{itemize}
\item This document covers only languages and scripts recognized in India.
\item This document covers all complex OFF features required for Indian scripts.
\item This document is based on the current expertise of community members 
working in this area.
\item This document is from the typography perspective for each script and many 
not be linguistically correct.
\item This document does not cover design or calligraphy style aspects but 
covers only technical aspects.
\item This document is based on Unicode 6.2 and ISO/IEC 14496-22:2009 (Second 
Edition) "Open Font Format" standard.
\item This document is not a tutorial on font design or development and does not 
teach typography.
\end{itemize}

\section{How to use this document}

Elaborate, Illustrated font design guideline document for Indic fonts by font 
designers, developers, language experts.
- designer freedom to adapt

\section{Notes on Collaboration}

\chapter{General Concepts}

\secstar{Complex script }

Complex script is writing system in which the shape or positining of a character
depends on its relation to other characters.

What makes it complex?
* Bi-directional: text is written right-to-left (For example: Arabic, Hebrew)
and left-to-right (For example: Devanagari)

* Contex sensitive shaping and ligatures: Character may change shape depending
upon position.

* Ordering:
In Gujarati, Ki is where 'i' is place before 'K'.

//FIXME
What is a complex script? What makes it complex, some examples, screenshots
2-3 paragraph + images
How it differs from simple scripts like Latin Character

\secstar{Glyph }

A glyph is an element of writing. It can be a single character or group of characters. 
Visually, if you see one or more characters form a single visual unit, it is called a glyph.

In typography, it is "the specific shape, design, or representation of a character".
\footnote{Ilene Strizver. "Confusing (and Frequently Misused) Type Terminology, Part 1". fonts.com. Monotype Imaging.}

It is a particular graphical representation, in a particular typeface, of an element of written language, which could be a grapheme, or part of a grapheme, or sometimes several graphemes in combination 
 
To illustrate this concept, a set glyphs inside a latin font and a Malayalam font as seen in fontforge is given below.

\begin{figure}[h]
    \centering
    \includegraphics[width=0.8\textwidth]{glyph-fontforge-roboto.png}
    \caption{Glyphs inside Roboto font}
\end{figure}

\begin{figure}[h]
    \centering
    \includegraphics[width=0.8\textwidth]{glyph-fontforge-meera.png}
    \caption{Glyphs inside Meera font}
\end{figure}

\begin{figure}[h]
    \centering
    \includegraphics[width=0.8\textwidth]{glyph-fontforge-lohit-bengali.png}
    \caption{Glyphs inside Lohit Bengali font}
\end{figure}

\secstar{Ligatures }

In writing and typography, a ligature occurs where two or more graphemes or letters are joined as a single glyph. Ligatures usually replace consecutive characters sharing common components and are part of a more general class of glyphs called "contextual forms", where the specific shape of a letter depends on context such as surrounding letters or proximity to the end of a line.
\footnote{\url{https://en.wikipedia.org/wiki/Typographic_ligature}}

By way of example, the common ampersand '\&' represents the Latin conjunctive word et, for which the English equivalent is the word "and". The ampersand's symbol is a ligature, joining the old handwritten Latin letters e and t of the word et, so that the word is represented as a single glyph.

The Brahmic abugidas make frequent use of ligatures in consonant clusters. The number of ligatures employed may be language-dependent; thus many more ligatures are conventionally used in Devanagari when writing Sanskrit than when writing Hindi. Having 37 consonants in total, the total number of ligatures that can be formed in Devanagari using only two letters is 1369, though few fonts are able to render all of them.

\begin{figure}[h]
   \centering
   {\hindi\textexample द्ध्र्य }
   \caption{The Devanagari ddhrya-ligature {\hindi (द् + ध् + र् + य = द्ध्र्य) } }
\end{figure}

\begin{figure}[h]
   \centering
   {\malayalam\textexample  ക്തു}
   \caption{The Malayalam kthu-ligature {\malayalam (ക + ് + ത + ു ) } }
\end{figure}

\begin{figure}[h]
  \centering
  {\gujarati\textexample સ +  ં = સં}
  \caption{The Gujarati 's' formed from SA and ANUSVARA}
\end{figure}

\secstar{Cluster/Syllable }

\secstar{Akhand }

\secstar{Ra Forms }
//FIXME
Explain Reph
Explain Rakar

\subsection*{Matra}
//FIXME
Explain prebase, postbase matra with examples and images

\secstar{Split Matra }

\secstar{Reordering }

\secstar{Zero Widh Joiner }

\secstar{Zero Width Non Joiner }

\secstar{Stacking }


\chapter{Open font format}

\section{Introduction}

\section{GPOS}

\section{GSUB}

\section{GDEF}

\section{Shaping Engine}

[Some conent from http://behdad.org/text/ can be used ]
\section{Shape Glyph sequence}

\section{Position Glyph sequence}

\section{Reference fonts}

To illustrate the concepts in this document, we will be using a set of reference fonts for each script, some times more than one per script. 
List of fonts table

\section{Reference Rendering Engine}

We are using Harfbuzz as  a reference rendering engine. Examples given are working perfectly with Harfbuzz but any rendering engine conforms to the Open font specification will give same result.


\chapter{Bengali}
\section{Introduction}

\section{Reference fonts}

\subsection{Lohit Bengali}

Lohit Bengali is one of the popular fonts for Bengali.

\subsection{History}
In 2004, Red Hat released five fonts for the Indian language under GPL 2.
The fonts were originally developed by Modular Infotech
\footnote{Modular Infotech \url{http://www.modular-infotech.com/}}.
In 2011, Red Hat relicensed fonts under SIL OFL 1.1 license
\footnote{License change announcement of Lohit fonts
\url{https://www.redhat.com/archives/lohit-devel-list/2011-September/msg00000.html}}.
The fonts named Lohit which means Red in Sanskrit. Currently, the font family
supports 21 Indian languages: Assamese, Bengali, Devanagari (Hindi, Kashmiri,
Konkani, Maithili, Marathi, Nepali, Sindhi, Santali, Bodo, Dogri), Gujarati,
Kannada, Malayalam, Manipuri, Oriya, Punjabi, Tamil, and Telugu.

Now, Fedora Project and its contributors took the responsibility to consolidate
the further efforts and improvements of the Lohit fonts.

Homepage: {\url{https://fedorahosted.org/lohit/}}

\chapter{Devanagari}
\section{Introduction}

Devanagari is an abugida alphabet for India and Nepal. Devanagari is used to
write Standard Hindi, Marathi, Nepali along with Awadhi, Bodo, Bhojpuri,
Gujari, Pahari, (Garhwali and Kumaoni), Konkani, Magahi, Maithili, Marwari,
Bhili, Newar, Santhali, Tharu, and sometimes Sindhi, Dogri, Sherpa, Kashmiri
and Punjabi. It was formerly used to write Gujarati. Because it is the
standardized script for the Hindi language, Devanagari is one of the most used
and adopted writing systems in the world.

\section{Reference fonts}

\subsection{Lohit Devanagari}
Lohit Devanagari font is consider as most popular Devanagari font in India.
Gargi, Chandas, Kalimati, Samanata, Nakula are other popular free and
opensource fonts available.

We will take Lohit Devanagari as reference font.

\subsection{History}
In 2004, Red Hat released five Indian language fonts as open source licensed
under the GPL. In 2011, Red Hat relicensed fonts under SIL OFL 1.1 license.
The fonts named Lohit which means Red in Sanskrit. Currently, the font family
supports 21 Indian languages: Assamese, Bengali, Devanagari (Hindi, Kashmiri,
Konkani, Maithili, Marathi, Nepali, Sindhi, Santhali, Bodo, Dogri), Gujarati,
Kannada, Malayalam, Manipuri, Odiya, Punjabi, Tamil, and Telugu.

Now, Fedora Project and its contributors took the responsibility to consolidate
the further efforts and improvements of the Lohit fonts.

Homepage: {\url{https://fedorahosted.org/lohit/}}

\chapter{Gujarati}
\secstar{Introduction}

Gujarati script is an abugida like many other Indic scripts rather than
alphabet. It is used for languages like Gujarati and Kutchi. Gujarati script is
variant of Devanagari script differentiated by the loss of the characteristic
horizontal line running above the letters and by a small number of
modifications in the remaining characters.

The modern Gujarati alphabet has 13 vowel letters, 36 consonant letters, 12
vowel extensions and few other symbols.

\secstar{Reference fonts}

We will use Lohit Gujarati as reference font as of now. Other popular Unicode
Gujarati fonts are: Shruti (non free, Microsoft), Rekha, Aakar and Kalapi.

\subsection{Lohit Gujarati}
// FIXME: Short introduction, designers, maintainers, usage info, popularity of
the font.

Lohit Gujarati font is consider as most popular Gujarati font.

\subsection{History}
In 2004, Red Hat released five Indian language fonts as open source licensed
under the GPL. In 2011, Red Hat relicensed fonts under SIL OFL 1.1 license.
The fonts named Lohit which means Red in Sanskrit. Currently, the font family
supports 21 Indian languages: Assamese, Bengali, Devanagari (Hindi, Kashmiri,
Konkani, Maithili, Marathi, Nepali, Sindhi, Santali, Bodo, Dogri), Gujarati,
Kannada, Malayalam, Manipuri, Oriya, Punjabi, Tamil, and Telugu.

Now, Fedora Project and its contributors took the responsibility to consolidate
the further efforts and improvements of the Lohit fonts.

Homepage: {\url{https://fedorahosted.org/lohit/}}


\input{Chapters/Kannada.tex}
\chapter{Malayalam}
\secstar{Introduction}

Like many other Indic scripts, it is an abugida, or a writing system
that is partially “alphabetic” and partially syllable-based. The
modern Malayalam alphabet has 13 vowel letters, 36 consonant letters,
and a few other symbols. The Malayalam script is a Vattezhuttu script,
which had been extended with Grantha script symbols to represent
Indo-Aryan loanwords. The script is also used to write several
minority languages such as Paniya, Betta Kurumba, and Ravula. The
Malayalam language itself was historically written in several
different scripts.

As is the case for many other Brahmi-derived scripts in the Unicode
Standard, Malayalam uses a virama character to form consonant
conjuncts. The virama sign itself is known as candrakala({\meera ചന്ദ്രക്കല}) in
Malayalam.

When the candrakala sign is visibly shown in Malayalam, it usually
indicates the suppression of the inherent vowel, but it sometimes
indicates instead a reduced schwa sound, often called “half-u” or
samvruthokaram. In the later case, there can also be a -u vowel sign,
and the base character can be a vowel letter. In all cases, the
candrakala sign is represented by the character U+0D4D malayalam sign
virama, which follows any vowel sign that may be present and precedes
any anusvara that may be present.

FIXME samvruthokaram needs more explanation, refer
{\url{http://thottingal.in/documents/Malayalam_Unicode_Report_of_Workshop_Kerala_University.pdf}}
and {\url{http://smc.org.in/doc/rachana-malayalam-collation.pdf}}

\secstar{Orthography variation}

Malayalam has two orthography variations, both actively
used. Generally they are known as Old orthography and Modern
orthography. Old orthography is also known as traditional
orthography. Modern orthography is also known as reformed orthography.

Old orthography is generally identified by large number of ligature
glyphs, often formed by more than one consonants or vowel
combination. For example, in old orthography, consonant {\meera ക} +
vowel { \meera ൂ} will form കൂ as a single ligature. similarly a
conjunct {\meera ക + ് + ത} will form {\meera ക്ത} as a single
ligature. Font developers of traditional stye fonts design and draw
this large set of glyphs. For a reference font Meera, the number of
glyphs is approximately over a thousad.

In the 1970s and 1980s, Malayalam underwent orthographic reform due to
printing difficulties. The treatment of the combining vowel signs u
and uu was simplified at this time. These vowel signs had previously
been represented using special cluster graphemes where the vowel signs
were fused beneath their consonants, but in the reformed orthography
they are represented by spacing characters following their consonants.


\begin{figure}[h]
   {\meera\large  മലയപ്പുലയനാ മാടത്തിൻമുറ്റത്തു\\
മഴ വന്ന നാളൊരു വാഴ നട്ടു.\\
മനതാരിലാശകൾപോലതിലോരോരോ \\
മരതകക്കൂമ്പു പൊടിച്ചുവന്നു.\\
അരുമക്കിടാങ്ങളിലൊന്നായതിനേയു -\\
മഴകിപ്പുലക്കള്ളിയോമനിച്ചു.}
   \caption{Text rendering using traditional orthography. Font used is Meera }
\end{figure}

\begin{figure}[h]
   {\lohitmalayalam\large  മലയപ്പുലയനാ മാടത്തിൻമുറ്റത്തു\\
മഴ വന്ന നാളൊരു വാഴ നട്ടു\\
മനതാരിലാശകൾപോലതിലോരോരോ \\
മരതകക്കൂമ്പു പൊടിച്ചുവന്നു\\
അരുമക്കിടാങ്ങളിലൊന്നായതിനേയു\\
മഴകിപ്പുലക്കള്ളിയോമനിച്ചു}
   \caption{Text rendering using modern orthography. Font used is Lohit Malayalam}
\end{figure}

The above examples give high level difference in orthograhy. 
Since both orthograhy is prominent in day to day usage of Malayalam, it will be helpful to get more
understanding about the orthography variation. So let us go some more deep into this differences.

For vowel signs, only u vowel({\meera ു}) and uu vowel({\meera ൂ})  make the difference. {\meera ു}
\begin{figure}[h]
   Traditional: \\{\textexample{ \meera ക + ു = കു  \\
ക + ൂ = കൂ } }\\
 Modern:\\ \textexample{ \raghumalayalam ക + ു = കു \\
ക + ൂ = കൂ }\\
   \caption{Rendering difference of u and uu vowel signs}
\end{figure}

As you can see, in traditional orthograhy the vowel sign attach to the consonant, while in modern script, the vowel sign is separate.

Reph sign veries in both orthographies. Reph sign is formed by VIRAMA+ RA ie {\malayalam ് + ര}.
\begin{figure}[h]
Traditional:\\ {\meera\textexample  പ്ര }\\
Modern: \\ {\lohitmalayalam\textexample  പ്ര }
   \caption{Rendering difference of u and uu vowel signs}
\end{figure}

As you can see, in traditional orthograhy the reph sign attach to the consonant, while in modern script,reph sign is separate and it get re-ordered to the left of consonant.


\secstar{Reference fonts}

Since we need to illustrate both orthographies we will be using 2
fonts. For old orthography, we will use Meera font and for modern
orthography we will use Lohit Malayalam.

\subsection {Meera Font}
{\meera മീര മാതൃക }
// FIXME: Short introduction, designers, maintainers, usage info, popularity of the font.

Meera font is maintained by Swathanthra Malayalam Computing initiative.

Homepage: {\url{https://savannah.nongnu.org/projects/smc}}

\subsection {Lohit Malayalam Font}
// FIXME: Short introduction, designers, maintainers, usage info, popularity of the font.

\subsection {History}
In 2004, Red Hat released five Indian language fonts as open source licensed
under the GPL. In 2011, Red Hat relicensed fonts under SIL OFL 1.1 license.
The fonts named Lohit which means Red in Sanskrit. Currently, the font family
supports 21 Indian languages: Assamese, Bengali, Devanagari (Hindi, Kashmiri,
Konkani, Maithili, Marathi, Nepali, Sindhi, Santali, Bodo, Dogri), Gujarati,
Kannada, Malayalam, Manipuri, Oriya, Punjabi, Tamil, and Telugu.

Now, Fedora Project and its contributors took the responsibility to consolidate
the further efforts and improvements of the Lohit fonts.

Homepage: {\url{https://fedorahosted.org/lohit/}}

\secstar{Technical details}
\subsection {Opentype Script Tags - mlym and mlm2}
\subsection {Reordering}
\subsection {Dotted Circle}
\subsection {Samvruthokaram}
\subsection {Reph}
\subsection {Dot Reph}
\subsection {Chillus}
\subsection {Conjunct Signs for യ, ര, ല}
\subsection {Stacking}
\subsection {Font metrics}
\subsection {Positioning rules}
\subsection {ZWNJ and ZWJ Signs}
\subsection {Prebase substitutions}
\subsection {Akhand forms}
\subsection {Below base forms}
\subsection {Below base substitutions}
\subsection {Half forms}
\subsection {Postbase forms}
\subsection {Latin glyphs and punctuations}
\subsection {Kerning}
\subsection {Shape references}
\subsection {Left and right bearings}
\subsection {Italic variant}
\subsection {Bold variant}

\input{Chapters/Oriya.tex}
\input{Chapters/Panjabi.tex}
\input{Chapters/Tamil.tex}
\chapter{Telugu}
\section{Introduction}

Telugu script, an abugida from the Brahmic family of scripts, is used to write the Telugu language, a language found in the South Indian state of Andhra Pradesh as well as several other neighboring states. It gained prominence during Vengi Chalukyan era. 

\input{Chapters/Glossary.tex}
\listoffigures
\chapter{References}

State of Text Rendering By Behdad Esfahbod <behdad behdad org>: {\url{http://behdad.org/text/}}

Pothana Paper: {\url{http://upload.wikimedia.org/wikipedia/te/c/c2/Pothanapaper.PDF}}

\input{Chapters/Appendices.tex}
\end{document}
